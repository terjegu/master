\chapter{Results} % (fold)
\label{cha:results}

some introduction...

\section{Pitch and voiced/unvoiced detection} % (fold)
\label{sec:pitch_and_voiced_unvoiced_detection}




% section Pitch and voiced/unvoiced detection (end)


\section{Fundamental Frequency Transformation} % (fold)
\label{sec:fundamental_frequency_ransformation}
The standalone fundamental frequency transform was tested in a controlled environment with the target cepstrum as input from the training set. The transformation used a 64 mixture GMM with full covariance matrices and was trained with 100 sentences where the unvoiced frames were discarded, yielding a total of 33000 training vectors. The results from the different setups are presented in Table~\ref{tab:pitch_pred_target_input}. \TODO{rerun with CC from training set}.
\begin{table}[htbp]
	\begin{center}
		\topcaption{Pitch prediction with 10 000 target cepstrum vectors as input.}
		\label{tab:pitch_pred_target_input}	
		\begin{tabular}{lrr}
			\toprule
			\multicolumn{1}{c}{\emph{Method}} & \multicolumn{1}{c}{\emph{Mean}} & \multicolumn{1}{c}{\emph{Std deviation}} \\
			\midrule
			Source $F_0$ & 0 Hz & 0 Hz \\
			No smoothing & 11.2 Hz & 10.3 Hz\\
			Moving average & 11.1 Hz & 10.1 Hz\\
			Smoothing & 10.7 Hz & 10.0 Hz \\
			\bottomrule			
		\end{tabular}		
	\end{center}
\end{table}
The results are not as impressive as in the research of T. En-Najjary \etal \cite{najjary03new}. However, there is an improvement compared to using the source speaker's fundamental frequency. 

\begin{figure}[htbp]
	\begin{center}
		\includegraphics[width=.9\textwidth]{fig/pitch_trans}
		\caption{Pitch transformation and different smoothing techniques with target $Y_{cc}$ as input.}
		\label{fig:pitch_trans}
	\end{center}
\end{figure}
Figure~\ref{fig:pitch_trans} depicts the transformed $F_0(t)$ for a test sentence. The transform without smoothing does not appear to be too bad from Table~\ref{tab:pitch_pred_target_input}, but Figure~\ref{fig:pitch_trans} reveals a fluctuating $F_0$ which is not very natural. The two smoothing techniques produce a $F_0$ contour which is similar to the correct contour, and in this particular case the moving average is the best choice. 

In a real scenario the available input to the $F_0$ transform is the transformed $\mathbf{\hat{X}}_{cc}$ vectors. 
\begin{table}[htbp]
	\begin{center}
		\topcaption{Pitch prediction error with transformed target vector as input.}
		\label{tab:pitch_pred_transformed_input}	
		\begin{tabular}{lrr}
			\toprule
			\multicolumn{1}{c}{\emph{Method}} & \multicolumn{1}{c}{$\mu_e$} & \multicolumn{1}{c}{$\sigma_e$}\\
			\midrule
			Source pitch & 14.1 Hz & 16.1 Hz\\
			No smoothing 32 & 11.4 Hz  & 12.4 Hz\\
			Moving average 64 & 9.4 Hz  & 14.0 Hz\\
			Smoothing 128 & 9.4 Hz & 13.7 Hz\\
			\bottomrule			
		\end{tabular}		
	\end{center}
\end{table}
This introduces another source of error and the transformation precision is degraded as shown in Table~\ref{tab:pitch_pred_transformed_input}. The table shows an average result for 10 000 transformations.

To relax the sources of error the fundamental frequency can be transformed together with the spectrum as a joint vector $\mathbf{z} = \sbrackets{\mathbf{y}_{cc},F_0}$. Table~\ref{tab:f0_joint_transform} shows a comparison to the standalone transformation with transformed cepstrum vectors as input and the joint transformation.
\begin{table}[htbp]
	\begin{center}
		\topcaption{Joint fundamental frequency and spectrum transformation}
		\label{tab:f0_joint_transform}
		\begin{tabular}{lll}
			\toprule
			\multicolumn{1}{c}{\emph{Method}} & \multicolumn{1}{c}{\emph{$\mu_e$}} & \multicolumn{1}{c}{\emph{$\sigma_e$}}\\
			\midrule
			Standalone & text & text \\
			Joint transform & text & text \\
			\bottomrule			
		\end{tabular}		
	\end{center}	
\end{table}
\TODO{comment}
% section Fundamental Frequency Transformation (end)

\section{Frequency Spectrum Transformation} % (fold)
\label{sec:frequency_transformation}
The frequency transformation was done in the cepstrum domain. A 128 mixture GMM with diagonal covariance matrices was trained with the same training set as the $F_0$ transformation. Four different setups were tested and the absolute results of the Itakura distance \eqref{eq:itakura_distance} and the $L_2$ metric \eqref{eq:l2_metric} are shown in Table~\ref{tab:absolute_results}.
\begin{table}[htbp]
	\begin{center}
		\topcaption{Absolute error in frequency transformation.}
		\label{tab:absolute_results}
		\begin{tabular}{lrrrr}
			\toprule
			\multicolumn{1}{c}{\emph{Setup}} & \multicolumn{1}{c}{\emph{Itakura pre}} & \multicolumn{1}{c}{\emph{Itakura post}} & \multicolumn{1}{c}{\emph{$L_2$ pre}} & \multicolumn{1}{c}{\emph{$L_2$ post}}\\
			\midrule
			Normal & 0 & 0 & 0 & 0 \\
			Pre-emphasis & 0.9677 & 0.5320 & 5.7336 & 4.0839 \\
			$c_0$ excluded & 0 & 0 & 0 & 0 \\
			Pre-emph and $c_0$ excluded & 0.9677 & 0.7227 & 5.7336 & 4.8453 \\
			\bottomrule			
		\end{tabular}		
	\end{center}	
\end{table}

A different point of view is the relative improvement to the starting point. The normalized results are shown in Table~\ref{tab:normalized_results} including the normalized cepstral distance.
\begin{table}[htbp]
	\begin{center}
		\topcaption{Normalized error in frequency transformation.}
		\label{tab:normalized_results}
		\begin{tabular}{lrrr}
			\toprule
			\multicolumn{1}{c}{\emph{Setup}} & \multicolumn{1}{c}{\emph{Itakura}} & \multicolumn{1}{c}{\emph{$L_2$}} & \multicolumn{1}{c}{\emph{NCD}}\\
			\midrule
			Normal & 0 & 0 & 0 \\
			Pre-emphasis & 0.5498 & 0.7123 & 0.5650 \\
			$c_0$ excluded & 0 & 0 & 0 \\
			Pre-emph and $c_0$ excluded & 0.7468 & 0.8451 & 0.62094 \\
			\bottomrule			
		\end{tabular}		
	\end{center}	
\end{table}

Table~\ref{tab:spectrum_joint_transform} shows a comparison on the spectrum transformation of joint and standalone transformation.
\begin{table}[htbp]
	\begin{center}
		\topcaption{Comparison of standalone and joint spectrum transformation}
		\label{tab:spectrum_joint_transform}
		\begin{tabular}{lll}
			\toprule
			\multicolumn{1}{c}{\emph{Method}} & \multicolumn{1}{c}{\emph{$\mu_e$}} & \multicolumn{1}{c}{\emph{$\sigma_e$}}\\
			\midrule
			Standalone & text & text \\
			Joint transform & text & text \\
			\bottomrule			
		\end{tabular}		
	\end{center}	
\end{table}
\TODO{comment}

\TODO{Claim of pre-emph improve high-freq transform by boosting high-freq energy. figures}


The setup with the best performance was the ... The average density of the LPC coefficients from one sentence, 500 vectors, are depicted in Figure~\ref{fig:hist_lpc}. \TODO{comment}
\begin{figure}[htbp]
	\begin{center}
	\subfigure[Source]
	{
		\includegraphics[width = .45\textwidth]{fig/hist_lp_source}
		\label{fig:hist_source}
	}
	\subfigure[Converted]
	{
		\includegraphics[width = .45\textwidth]{fig/hist_lp_source}
		\label{fig:hist_converted}
	}
	\subfigure[Target]
	{
		\includegraphics[width = .45\textwidth]{fig/hist_lp_source}
		\label{fig:hist_target}
	}
	\caption{Average density of LPC coefficients, excluding $\alpha_0$}
	\label{fig:hist_lpc}
	\end{center}
\end{figure}


% section Frequency Transformation (end)
% chapter Results (end)