\chapter{Results} % (fold)
\label{cha:results}

% some introduction...
% 
% \section{Pitch and voiced/unvoiced detection} % (fold)
% \label{sec:pitch_and_voiced_unvoiced_detection}
% 
% 
% 
% 
% % section Pitch and voiced/unvoiced detection (end)


\section{Fundamental Frequency Transformation} % (fold)
\label{sec:fundamental_frequency_ransformation}
The standalone fundamental frequency transform was tested in a controlled environment with the target cepstrum as input from the training set. The transformation used a 64 mixture GMM with full covariance matrices and was trained with 100 sentences where the unvoiced frames were discarded, yielding a total of 33000 training vectors. The results from the different setups are presented in Table~\ref{tab:pitch_pred_target_input}.
\begin{table}[htbp]
	\begin{center}
		\topcaption{$F_0$ transformation with target cepstrum vectors as input for 10 sentences.}
		\label{tab:pitch_pred_target_input}	
		\begin{tabular}{lrr}
			\toprule
			\multicolumn{1}{c}{\emph{Method}} & \multicolumn{1}{c}{\emph{Mean}} & \multicolumn{1}{c}{\emph{Std deviation}} \\
			\midrule
			Source $F_0$ & 24.82 Hz & 23.38 Hz \\
			No smoothing & -0.75 Hz & 20.31 Hz\\
			Moving average & -0.77 Hz & 19.71 Hz\\
			Improved m-avg & -1.58 Hz & 19.19 Hz \\
			Delta limit & -1.71 Hz & 19.93 Hz \\
			\bottomrule			
		\end{tabular}		
	\end{center}
\end{table}

% \begin{table}[htbp]
% 	\begin{center}
% 		\topcaption{$F_0$ transformation with target cepstrum vectors as input for 10 sentences.}
% 		\label{tab:pitch_pred_target_input}	
% 		\begin{tabular}{lrr}
% 			\toprule
% 			\multicolumn{1}{c}{\emph{Method}} & \multicolumn{1}{c}{\emph{Mean}} & \multicolumn{1}{c}{\emph{Std deviation}} \\
% 			\midrule
% 			Source $F_0$ & 29.32 Hz & 17.40 Hz \\
% 			No smoothing & 15.19 Hz & 13.50 Hz\\
% 			Moving average & 14.77 Hz & 13.06 Hz\\
% 			Improved m-avg & 14.69 Hz & 12.45 Hz \\
% 			Delta limit & 14.69 Hz & 12.45 Hz \\
% 			\bottomrule			
% 		\end{tabular}		
% 	\end{center}
% \end{table}

The results are close to the results of T. En-Najjary \etal \cite{najjary03new}. Moving average and no smoothing at all appears to be the best choice.

\begin{figure}[htbp]
	\begin{center}
		\includegraphics[width=.9\textwidth]{fig/pitch_trans}
		\caption{Contour of $F_0$ transformation with different smoothing techniques and target $Y_{cc}$ as input.}
		\label{fig:pitch_trans}
	\end{center}
\end{figure}
Figure~\ref{fig:pitch_trans} depicts the transformed $F_0(t)$ for a test sentence. The transform without smoothing does not appear to be too bad from Table~\ref{tab:pitch_pred_target_input}, but Figure~\ref{fig:pitch_trans} reveals a fluctuating $F_0$ which is not very natural. The two smoothing techniques produce a $F_0$ contour which is similar to the correct contour, and in this particular case the moving average is the best choice. 

In a real scenario the available input to the $F_0$ transform is the transformed $\mathbf{\hat{X}}_{cc}$ vectors. 
\begin{table}[htbp]
	\begin{center}
		\topcaption{$F_0$ transformation error with transformed cepstral vector as input for 10 sentences.}
		\label{tab:pitch_pred_transformed_input}	
		\begin{tabular}{lrr}
			\toprule
			\multicolumn{1}{c}{\emph{Method}} & \multicolumn{1}{c}{$\mu_e$} & \multicolumn{1}{c}{$\sigma_e$}\\
			\midrule
			Source pitch & 24.82 Hz & 23.38 Hz\\
			No smoothing & -0.05 Hz  & 26.29 Hz\\
			Moving average & -0.08 Hz  & 24.57 Hz\\
			Improved m-avg & 1.18 Hz & 23.59 Hz\\
			Delta limit & 1.25 Hz & 24.56 Hz \\
			\bottomrule			
		\end{tabular}		
	\end{center}
\end{table}
% \begin{table}[htbp]
% 	\begin{center}
% 		\topcaption{$F_0$ transformation error with transformed cepstral vector as input for 10 sentences.}
% 		\label{tab:pitch_pred_transformed_input}	
% 		\begin{tabular}{lrr}
% 			\toprule
% 			\multicolumn{1}{c}{\emph{Method}} & \multicolumn{1}{c}{$\mu_e$} & \multicolumn{1}{c}{$\sigma_e$}\\
% 			\midrule
% 			Source pitch & 29.32 Hz & 17.40 Hz\\
% 			No smoothing & 19.89 Hz  & 17.19 Hz\\
% 			Moving average & 18.96 Hz  & 15.63 Hz\\
% 			Improved m-avg & 18.52 Hz & 14.66 Hz\\
% 			Delta limit & 0 Hz & 0 Hz \\
% 			\bottomrule			
% 		\end{tabular}		
% 	\end{center}
% \end{table}
This introduces another source of error and the transformation precision is degraded as shown in Table~\ref{tab:pitch_pred_transformed_input}. The table shows an average result from 10 sentences.

To relax the sources of error the fundamental frequency can be transformed together with the spectrum as a joint vector $\mathbf{z} = \sbrackets{\mathbf{y}_{cc},F_0}$. Table~\ref{tab:f0_joint_transform} shows a comparison to the standalone transformation with transformed cepstrum vectors as input and the joint transformation.
\begin{table}[htbp]
	\begin{center}
		\topcaption{Joint fundamental frequency and spectrum transformation}
		\label{tab:f0_joint_transform}
		\begin{tabular}{lll}
			\toprule
			\multicolumn{1}{c}{\emph{Method}} & \multicolumn{1}{c}{\emph{$\mu_e$}} & \multicolumn{1}{c}{\emph{$\sigma_e$}}\\
			\midrule
			Standalone & -0.05 & 26.29 \\
			Joint transform & -3.43 & 22.26 \\
			\bottomrule			
		\end{tabular}		
	\end{center}	
\end{table}
\TODO{comment}

The transformation in hand is clearly not context dependent. If the context was known the $F_0$ contour could be scaled, with \eg a parabola for a typical statement, to enhance the performance. Without context information this is impossible since the prosody in a statement is clearly different than in a question. The source $F_0$ could be used in the transformation to introduce some context awareness.
% section Fundamental Frequency Transformation (end)

\section{Frequency Spectrum Transformation} % (fold)
\label{sec:frequency_transformation}
The frequency transformation was done in the cepstrum domain. A 128 mixture GMM with diagonal covariance matrices was trained with the same training set as the $F_0$ transformation. Four different setups were tested and the average results of the Itakura distance \eqref{eq:itakura_distance}, the $L_2$ metric \eqref{eq:l2_metric} and the cepstral distance \eqref{eq:cepstral_distance} of 10 sentences are shown in Table~\ref{tab:absolute_results}. The distance of the source and target feature vectors before transformation are; itakura = 0.6053, $L_2$ = 4.4985 and CD = 0.4483.
\begin{table}[htbp]
	\begin{center}
		\topcaption{Absolute error in frequency transformation.}
		\label{tab:absolute_results}
		\begin{tabular}{lrrr}
			\toprule
			\multicolumn{1}{c}{\emph{Setup}} & \multicolumn{1}{c}{\emph{Itakura}} & \multicolumn{1}{c}{\emph{$L_2$}} & \multicolumn{1}{c}{\emph{CD}}\\
			\midrule
			Normal &  0.4565 & 3.6852 & 0.2939 \\
			Pre-emphasis & 0.5162 & 3.9512 & 0.3203 \\
			$c_0$ excluded & 0.5111 & 3.9790 & 0.3361 \\
			Pre-emph, $c_0$ excluded & 0.6016 & 4.3725 & 0.3447 \\
			\bottomrule			
		\end{tabular}		
	\end{center}	
\end{table}

% \begin{table}[htbp]
% 	\begin{center}
% 		\topcaption{Absolute error in frequency transformation.}
% 		\label{tab:absolute_results}
% 		\begin{tabular}{lrrrr}
% 			\toprule
% 			\multicolumn{1}{c}{\emph{Setup}} & \multicolumn{1}{c}{\emph{Itakura pre}} & \multicolumn{1}{c}{\emph{Itakura post}} & \multicolumn{1}{c}{\emph{$L_2$ pre}} & \multicolumn{1}{c}{\emph{$L_2$ post}}\\
% 			\midrule
% 			Normal & 0.6053 & 0.4652 & 4.4985 & 3.7524 \\
% 			Pre-emphasis & 0.7925 & 0.5127 & 5.2366 & 3.9638 \\
% 			$c_0$ excluded &  0.6053 & 0.5267 & 4.4985 & 4.1227 \\
% 			Pre-emph and $c_0$ excluded & 0.7925 & 0.6038 & 5.2366 & 4.4277 \\
% 			\bottomrule			
% 		\end{tabular}		
% 	\end{center}	
% \end{table}


A different point of view is the relative improvement to the starting point. The normalised results are shown in Table~\ref{tab:normalized_results} for the same 10 sentences. By implementing pre-emphasis the source and target spectra are also affected which does not justify the absolute results of the pre-emphasis transform. The relative distance measures is therefor a better metric in this regard.
\begin{table}[htbp]
	\begin{center}
		\topcaption{Normalised error in frequency transformation.}
		\label{tab:normalized_results}
		\begin{tabular}{lrrr}
			\toprule
			\multicolumn{1}{c}{\emph{Setup}} & \multicolumn{1}{c}{\emph{Itakura$_N$}} & \multicolumn{1}{c}{\emph{N-L$_2$}} & \multicolumn{1}{c}{\emph{NCD}}\\
			\midrule
			Normal & 0.7542 & 0.8174 & 0.6553 \\
			Pre-emphasis &  0.6514 & 0.7545 & 0.5780 \\
			$c_0$ excluded & 0.8444 & 0.8826 & 0.7495 \\
			Pre-emph and $c_0$ excluded & 0.7591 & 0.8350 & 0.6219 \\
			\bottomrule			
		\end{tabular}		
	\end{center}	
\end{table}

Table~\ref{tab:spectrum_joint_transform} shows a comparison on the spectrum transformation of joint and standalone transformation.
\begin{table}[htbp]
	\begin{center}
		\topcaption{Comparison of standalone and joint spectrum transformation}
		\label{tab:spectrum_joint_transform}
		\begin{tabular}{lrrr}
			\toprule
			\multicolumn{1}{c}{\emph{Method}} & \multicolumn{1}{c}{\emph{Itakura}} & \multicolumn{1}{c}{\emph{$L_2$}} & \multicolumn{1}{c}{\emph{CD}}\\
			\midrule
			Standalone & 0.4565 & 3.6852 & 0.2939 \\
			Joint transform & 3.2840 & 10.157 & 3.0131 \\
			\bottomrule			
		\end{tabular}		
	\end{center}	
\end{table}
The joint transform did not have a dramatic effect on the $f_0$ transform, in fact it had a lower standard deviation than the standalone transform. But the joint transformation of the spectrum is even worse than no transformation at all. Of course the joint transformation could be used only for the pitch and spectrum could be transform in a separate transform, but since there was nothing to gain in the $f_0$ transform either there is not point in combining these two transformation in one.

\begin{figure}[htbp]
	\begin{center}
	\subfigure[Without pre-emphasis]
	{
		\includegraphics[width = .45\textwidth]{fig/freq_100}
		\label{fig:freq_100}
	}
	\subfigure[Pre-emphasis applied]
	{
		\includegraphics[width = .45\textwidth]{fig/freq_100_pre}
		\label{fig:freq_100_pre}
	}
	\subfigure[Without pre-emphasis]
	{
		\includegraphics[width = .45\textwidth]{fig/freq_125}
		\label{fig:freq_125}
	}
	\subfigure[Pre-emphasis applied]
	{
		\includegraphics[width = .45\textwidth]{fig/freq_125_pre}
		\label{fig:freq_125_pre}
	}
	\caption{The effect of pre-emphasis on magnitude spectrum.}
	\label{fig:the_effect_of_pre_emphasis_on_magnitude_spectrum_}
	\end{center}
\end{figure}


The setup with the best performance was the ... The average density of the LPC coefficients, excluding the first which was 1 for all vectors, from 10 sentences, yielding 28 000 parameters, are depicted in Figure~\ref{fig:hist_lpc}.
\begin{figure}[htbp]
	\begin{center}
		\includegraphics[width=.8\textwidth]{fig/lp_hist}
		\caption{Comparison of LP parameters distributions.}
		\label{fig:hist_lp}
	\end{center}
\end{figure}
The source and target LP vectors as a pretty even distribution of parameters, while the converted vectors are a little off. Even though they have the same mean value, the converted parameters has a more flat distribution and are dense in different regions than the target distribution.
% section Frequency Transformation (end)

\section{Listening Test} % (fold)
\label{sec:listening_test}

% section Listening Test (end)
% chapter Results (end)